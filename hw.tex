\documentclass{article}
\usepackage[utf8]{inputenc}
\usepackage[russian]{babel}

\title{Математический анализ}
\author{Лапутин Богдан}
\date{Апрель 2022}

\begin{document}

\maketitle

\paragraph{\textbf{Задание:}}
\[\oint\limits_{C}\mathrm{\frac{e^z}{(z-i)^2\cdot(z+2)}dz}\]

\begin{enumerate}
\item \(C=C_1: |\mathrm{z-i} |=2\)
\item \(C=C_2: \mathrm{|z+2-i|}=3\) 
\end{enumerate}

\textbf{Решение: }
	Находим корни знаменателя подынтегрального выражения:
\[\mathrm{(z-i)^2\cdot(z+2)}  = 0 \rightarrow  \mathrm{z=i}\;,\:\mathrm{z=-2}\]

1.\(\mathrm{D} := \{z : |z-i|  < 2\}\) содержит только один из корней (\(z = i\)), 
тогда интеграл вычисляется следующим образом:

\[\oint\limits_{C_1}\mathrm{\frac{e^z}{(z-i)^2\cdot(z+2)}dz} = \oint\limits_{\mathrm{|z-i|}=2}\mathrm{\frac{e^z}{(z-i)^2\cdot(z+2)}dz} = [\frac{e^z}{(z-i)^2\cdot(z+2)}=\frac{\frac{e^z}{z+2}}{(z-i)^2}]==\oint\limits_{\mathrm{|z-i|}=2}\mathrm{\frac{\frac{e^z}{z+2}}{(z-i)^2}dz}\]

Используя интегральную формулу Коши для производных, 
мы получаем:


\[\oint\limits_{\mathrm{|z-i|}=2}\mathrm{\frac{\frac{e^z}{z+2}}{(z-i)^2}dz }= \frac{2 \pi i}{1!}\cdot (\frac{e^z}{z+2})'|_{\mathrm{z=i}}=2 \pi i \cdot \frac{e^z\cdot (z+1)}{(z+2)^2}|_{\mathrm{z=i}}=2 \pi i \cdot \frac{e^i\cdot (i+1)}{(i+2)^2}\]


2.\(\mathrm{D} := \{z : \mathrm{|z+2-i|}<3\}\) содержит оба корня (\(z = i\; , \;z=-2\)) следовательно:

\[\oint\limits_{C}\mathrm{\frac{e^z}{(z-i)^2\cdot(z+2)}dz} = \oint\limits_{\mathrm{l_1}}\mathrm{\frac{e^z}{(z-i)^2\cdot(z+2)}dz}+ \oint\limits_{\mathrm{l_2}}\mathrm{\frac{e^z}{(z-i)^2\cdot(z+2)}dz}\]

Где \(\mathrm{l_1}\) и \(\mathrm{l_2}\) контуры, содержащие один из корней.
Пусть  \(\mathrm{l_1}\) это \(|z - i| = 1\), \(\mathrm{l_2}\) это \(|z+2| = 1\). Тогда:

\[\oint\limits_{C}\mathrm{\frac{e^z}{(z-i)^2\cdot(z+2)}dz} = \oint\limits_{\mathrm{|z - i| = 1}}\mathrm{\frac{\frac{e^z}{z+2}}{(z-i)^2}dz}+ \oint\limits_{\mathrm{l_2}}\mathrm{\frac{\frac{e^z}{(z-i)^2}}{z-(-2)}dz} = \frac{2 \pi i}{1!}\cdot (\frac{e^z}{z+2})'|_{\mathrm{z=i}} + \frac{2 \pi i}{0!}\cdot (\frac{e^z}{(z-i)^2})|_{\mathrm{z=-2}} \]

\[
    = 2 \pi i \cdot \frac{e^i\cdot (i+1)}{(i+2)^2} + 2 \pi i \cdot \frac{e^-2}{(i+2)^2} = 
        \frac{2 \pi i}{(i+2)^2}\cdot(e^-2 + e^i\cdot(i+1))
\]


\end{document}
